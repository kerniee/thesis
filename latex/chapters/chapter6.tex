\chapter{Conclusion}
\label{ch:conclusion}

This thesis has presented a comprehensive study on the integration of Combinatorial Testing (CT) and Metamorphic Testing (MT) for automated security testing of web applications. The proposed implementation of the COMER framework combines the strengths of CT and MT to generate test cases that can effectively identify vulnerabilities in complex cybersecurity software.

The evaluation of the COMER framework has demonstrated its effectiveness in generating test cases that can detect vulnerabilities in web applications. The results show that the COMER framework can generate test cases more efficiently than traditional CT methods, with a higher detection rate and similar generation time. However, preliminary tests of the current implementation showed that the execution time in some cases doubled compared to the traditional CT framework due to the increase in the number of test cases generated.

The study has also demonstrated the potential of the COMER framework to be standardized into a framework or toolset for automated security testing. The framework's ability to efficiently identify vulnerabilities and generate test cases makes it a valuable tool for cybersecurity professionals seeking to enhance the security of their applications.

However, the study has also highlighted the need for more research in the area of systematic generation of Metamorphic Relations (MRs) for complex cybersecurity software. The manual crafting of MRs for each specific test case is a time-consuming and labor-intensive process that requires domain expertise. The development of methodologies or tools that can automate the generation of MRs would significantly enhance the effectiveness of the COMER framework.

\section{Future Work}

The study has identified several areas for future research, including:

\begin{itemize}
\item Development of methodologies or tools that can automate the generation of Metamorphic Relations (MRs) for complex cybersecurity software.
\item Investigation of the application of the COMER framework to other domains, such as mobile applications, IoT devices, other Cybersecurity vulnerabilities and domains.
\item Exploration of the use of machine learning and artificial intelligence techniques to enhance the effectiveness of the COMER framework.
\item Further improvements to current implementation to reduce the number of test cases generated and improve execution time.
\item Compare with other state-of-the-art tools and frameworks for automated security testing.
\end{itemize}

These areas of research have the potential to further enhance the effectiveness of the COMER framework and to expand its applicability to other domains.
