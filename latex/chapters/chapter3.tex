%!TEX root = ../thesis.tex

\chapter{Methodology}
\label{ch:met}

This study employs the COMER methodology for the generation of test cases.
In order to assess the efficacy of the framework and conduct comparative analyses with other existing frameworks, I undertook the task of implementing it from scratch using the Python programming language.
The initial implementation by Niu \textit{et al.} \cite{comer} was coded in Java.
However, this implementation presented certain limitations; the code was compiled into a jar file, making it challenging to extend or modify functionalities.
Consequently, I opted to develop my own implementation in Python, making necessary modifications to facilitate experimentation with different datasets and enhancing the usability of the framework.
Python was chosen due to its user-friendly syntax and the availability of extensive libraries for data manipulation and visualization.


\section{COMER Framework}\label{sec:comer-framework}

The COMER framework serves the purpose of generating test cases, leveraging both traditional Combinatorial Testing (CT) methods and incorporating metamorphic relations during the generation process.
Upon receiving inputs including a set of metamorphic relations, a set of constraints, and abstract input parameters, COMER generates test cases that adhere to the specified constraints while satisfying certain metamorphic relations.

The framework operates in two primary stages:

\subsection{Abstract Test Case Generation}\label{subsec:abstract-test-case-generation}

The algorithm for abstract test case generation encompasses two main pathways.
At each step, there exists a probability distribution determining the choice between two flows.
In the first flow, the algorithm functions akin to a conventional CT generation algorithm, employing a greedy approach to generate test cases until achieving the desired t-way coverage.
Conversely, the second flow involves the selection of a previously generated test case, followed by the generation of subsequent test cases utilizing metamorphic relations.

\subsection{Concrete Test Case Generation}\label{subsec:concrete-test-case-generation}

Subsequently, the generated abstract test cases are mapped to concrete ones.
For test cases generated via CT methods, the concrete test case is produced by randomly selecting a value from the domain corresponding to the abstract value.
Conversely, for test cases generated through metamorphic relations, the framework utilizes the specified metamorphic relation to generate subsequent test cases.

\begin{figure}[hbt]
    \centering
    \includegraphics[]{figs/test_cases.png}
    \caption{Example of generating test cases for \textit{FindClosest} function. Note that $T1$ and $T2$ are parameters generated by utilizing MR.}
    \label{fig:secex}
\end{figure}


\section{Limitations}\label{sec:limitations}

Despite the advancements made in the original COMER implementation, certain limitations were identified.
Firstly, the reliance on a greedy approach for generating Combinatorial Testing (CT) test cases posed a notable drawback.
While this method can increase the speed of generation process, it may result in an excessive number of test cases, potentially compromising efficiency and resource utilization.
Secondly, the focus of the implementation solely on metamorphic relations with a single input and a single follow-up test case restricted its applicability to scenarios involving more complex MRs. Thirdly, the lack of provision for generating test cases for functions beyond those already supported in the implementation posed a significant constraint.

In my implementation, efforts were made to mitigate some of these limitations.
To overcome the limitation of supporting only a limited range of functions, I worked to improve the framework by allowing the generation of test cases for any function.
Additionally, in response to concerns regarding the effectiveness of the greedy approach, provisions were made to incorporate alternative algorithms for CT test case generation, thereby offering users flexibility in selecting the most suitable approach.
However, it is important to note that the limitation pertaining to MRs with multiple inputs and follow-up test cases remains unaddressed in my implementation, representing an area for potential future research and improvement.
