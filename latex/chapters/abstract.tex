\begin{abstract}

The rapid development of software and its increasing integration into various aspects of our daily lives have brought about unprecedented challenges in ensuring its security. With the convergence of the non-IT and IT industries, the number of software errors and vulnerabilities has increased, posing significant risks to individuals, organizations, and society as a whole. Despite continuous efforts to improve cybersecurity, the omnipresence of cyber threats and the lack of a single foolproof method to ensure software correctness persist as pressing concerns.

This thesis project aimed to address the critical need for enhanced cybersecurity by investigating the potential of Metamorphic Testing and Combinatorial Testing as effective techniques to identify and mitigate software vulnerabilities. The research sought to evaluate the combined use of these methods and evaluate their efficiency, effectiveness, and comprehensive coverage in detecting security flaws.

To achieve the research objective, a rigorous study design was employed, focusing on the application of Metamorphic Testing and Combinatorial Testing to diverse software systems and scenarios. The study design incorporated various black-box testing techniques, including Domain Testing, Boundary Value Analysis, and the utilization of Metamorphic Relations to address the Test Oracle Problem. Extensive experimentation and analysis were performed to assess the benefits and limitations of this combined approach.

The research project yielded substantial findings, demonstrating that the integration of Metamorphic and Combinatorial Testing techniques significantly improved the efficiency and quality of software security testing. The approach exhibited superior coverage of potential vulnerabilities and offered a more comprehensive understanding of software behavior under various conditions.

The contribution of this work lies in providing a robust framework for improving cybersecurity through the synergistic use of Metamorphic and Combinatorial Testing. The results have direct applicability in industry and academia, offering a practical strategy to identify and address software vulnerabilities more effectively. This research contributes to advancing the state-of-the-art in cybersecurity and provides valuable insights into safeguarding the integrity and reliability of software systems in an increasingly interconnected world.

\end{abstract}
