\documentclass[oneside,final,14pt,a4paper]{extreport}

\usepackage{tempora}

\usepackage{vmargin}
\setpapersize{A4}
\setmarginsrb{2.5cm}{2.2cm}{2.2cm}{2.2cm}{0pt}{10mm}{0pt}{13mm}
\usepackage{setspace}
\sloppy
\setstretch{1.5}
\usepackage{indentfirst}
\parindent=1.25cm

%%%%% ADDED TO SUPPORT TT BOLD FACES %%%%
\DeclareFontShape{OT1}{cmtt}{bx}{n}{<5><6><7><8><9><10><10.95><12><14.4><17.28><20.74><24.88>cmttb10}{}
\renewcommand{\ttdefault}{pcr}
%%%%% END %%%%%%%%%%%%%%%%%%%%%%%%%%%%%%%
\usepackage{atbegshi,picture}
\usepackage[T1,T2A]{fontenc}
\usepackage[utf8]{inputenc}
\usepackage[main=russian,english]{babel}
\usepackage[backend=biber,style=ieee,autocite=inline]{biblatex}
\bibliography{thesis.bib}
\usepackage{csquotes}
\usepackage{blindtext}


\usepackage{pdfpages}
\newenvironment{bottompar}{\par\vspace*{\fill}}{\clearpage}

% \usepackage{cite}
\usepackage{amsmath,amsfonts}
\usepackage{amsthm}
\newtheorem{theorem}{Theorem}
\newtheorem{corollary}{Corollary}
\newtheorem{lemma}{Lemma}
\newtheorem{proposition}{Proposition}
\theoremstyle{definition}
\newtheorem{definition}{Definition}
\theoremstyle{remark}
\newtheorem*{remark}{Remark}
\theoremstyle{remark}
\newtheorem*{example}{Example}



\usepackage{graphicx}
\graphicspath{{figs/}} %path to images
\usepackage{multirow,array}
\usepackage{caption}
\usepackage{subcaption}
\usepackage[unicode]{hyperref}
\hypersetup{colorlinks=true, linkcolor=black, citecolor=black}
\usepackage{paralist}
\usepackage{listings}
\usepackage{zed-csp}
\usepackage{fancyhdr}
\usepackage{color,colortbl}
\usepackage{booktabs}
\usepackage{epsfig} % for postscript graphics files

\usepackage{upgreek}
\usepackage{bm}
\usepackage{hyperref}
\usepackage{longtable}
\usepackage[font=singlespacing, labelfont=bf]{caption}
\usepackage{floatrow}

\pagestyle{fancyplain}

% remember section title
\renewcommand{\chaptermark}[1]%
	{\markboth{\chaptername~\thechapter~--~#1}{}}

% subsection number and title
\renewcommand{\sectionmark}[1]%
	{\markright{\thesection\ #1}}

\rhead[\fancyplain{}{\bf\leftmark}]%
      {\fancyplain{}{\bf\thepage}}
\lhead[\fancyplain{}{\bf\thepage}]%
      {\fancyplain{}{\bf\rightmark}}
\cfoot{} %bfseries


\newcommand{\dedication}[1]
   {\thispagestyle{empty}

   \begin{flushleft}\raggedleft #1\end{flushleft}
}

\DeclareFixedFont{\ttb}{T1}{txtt}{bx}{n}{12} % for bold
\definecolor{deepblue}{rgb}{0,0,0.5}
\definecolor{deepurple}{rgb}{0.5,0,0.3}
\definecolor{deepgreen}{rgb}{0,0.5,0}
\definecolor{halfgray}{RGB}{250, 250, 250}

\lstset{
	language=Python,
	columns=fullflexible,
	tabsize=4,
	keywordstyle=\ttb\color{deepblue},
	emph={self},
	emphstyle=\color{deepurple},
	keywordstyle=\ttb\color{deepblue},
	stringstyle=\color{deepgreen},
	showstringspaces=false,
 	backgroundcolor=\color{halfgray},
 	frame=tb,
 	xleftmargin=5mm,
 	aboveskip=5mm,
 	framexleftmargin=5mm,
}

\begin{document}

\includepdf[pages=-, offset=75 -75]{title.pdf}

\newpage
\tableofcontents
\begin{abstract}

The rapid development of software has increased security challenges and vulnerabilities, posing significant risks. Despite efforts, cyber threats persist, and no foolproof method ensures software correctness.

This thesis investigates Metamorphic Testing and Combinatorial Testing to improve cybersecurity. The research evaluates their combined effectiveness in detecting security flaws.

A rigorous study applied these testing methods to various software systems, using black-box techniques like Domain Testing, Boundary Value Analysis, and Metamorphic Relations. Extensive experimentation assessed their benefits and limitations.

Findings show that integrating these testing techniques significantly improves software security testing. This approach offers superior vulnerability coverage and a better understanding of software behavior.

The research provides a practical framework for enhancing cybersecurity, applicable in both industry and academia. It advances cybersecurity and offers insights into protecting software systems in an interconnected world.

\end{abstract}

\setcounter{page}{4}
% set manually the number, from which Глава 1 starts!
% Why do we put 4 in this case?
% Title page - page 1
% Оглавление - page 2
% Аннотация - page 3
% Глава 1 - page 4
% In your annotation the counter number can be different, please count carefully and insert the corresponding number.

\chapter{Introduction}
\label{chap:intro}

\section{Motivation}

In recent years, the rapid evolution of software development has resulted in a significant surge in software errors and vulnerabilities. The escalating number of Common Vulnerabilities and Exposures (CVEs) reported each year reflects the growing challenges faced in ensuring the security and reliability of software systems \cite{CVE}. As the digital landscape becomes increasingly complex and interconnected, the need for robust cybersecurity measures has become paramount.

The traditional approaches to software testing are often inadequate in detecting and mitigating vulnerabilities effectively. With the adoption of Continuous Integration practices as a standard in the industry, there's a heightened demand for automated testing methods that can seamlessly integrate into the development workflow. The primary objective is to identify and address bugs and vulnerabilities at the earliest stages of the software development life cycle.

Several techniques have been employed to enhance software testing and security, including but not limited to code analysis, fuzzing, and domain testing. While these methods have proven to be valuable, there remains a pressing need for more efficient and effective approaches to bolster cybersecurity measures.

\section{Proposed Approach}

This thesis proposes a novel methodology that combines two distinct testing techniques: Metamorphic Testing and Combinatorial Testing. The proposed approach adopts a semi-automated methodology for generating test cases. By leveraging both metamorphic and Combinatorial Testing principles, the aim is to streamline the testing process while ensuring adequate coverage of the system under test. This hybrid approach strikes a balance between manual intervention and automated testing, optimizing the efficiency and effectiveness of the testing process.

One of the key advantages of the proposed method is its ability to achieve comprehensive test coverage with a relatively small number of test cases. Combinatorial Testing, by systematically exploring the interaction of input parameters, helps in reducing the number of test cases required. Meanwhile, Metamorphic Testing enhances the efficiency of the testing process by leveraging the inherent properties of the system.

\section{Terminology}

This section presents an overview of key concepts in the field of metamorphic and Combinatorial Testing, focusing on their applications in cybersecurity. The definitions below lay the groundwork for understanding the methodologies discussed in subsequent sections.

\subsection{Black Box Testing}

\textbf{Black Box Testing} is a methodology where the internal workings or implementation details of the system under test are not known or considered by the tester. Testing is based solely on defined specifications or thought specs. Black-box testing relies only on the input/output behavior of the software \cite{Testing}.

Some examples of black-box testing techniques include:
\begin{itemize}
    \item Metamorphic Testing
    \item Combinatorial Testing
    \item Fuzz Testing
    \item Boundary Value Analysis
\end{itemize}

\subsection{Metamorphic Testing}

\textbf{Metamorphic Testing} is an effective technique for alleviating the oracle problem, testing "untestable" programs where failures are not revealed by checking individual outputs, but by checking expected relations among multiple executions of the program under test \cite{CybersecurityMT}.

\textbf{Metamorphic Relation (MR)} is a property or characteristic that should be maintained between multiple test case outputs. If an MR is violated, it suggests a potential defect in the software \cite{MetamorphicTestingReview}.

\textbf{Metamorphic Group (MG)} is a multiple sequence of inputs related to each other by a particular metamorphic relation. It consists of a sequence of source inputs and follow-up inputs \cite{MetamorphicTestingReview}. \textit{Example:} \textit{MR1: $f(x, y) = f(y, x)$} is a metamorphic relation, and \textit{MG1: {1, 2} $\rightarrow$ {2, 1}} is a metamorphic group.

\subsection{Combinatorial Testing}

\textbf{Combinatorial Testing} is a software testing technique that focuses on testing subsets of combinations of input values and preconditions to uncover defects. It's particularly effective when the number of parameters and their possible values is too large for exhaustive testing \cite{FELDERER20161}.

Consider a function $f(a, b, c)$ with boolean parameters.

\textbf{Test Case} is a set of conditions or variables under which a tester will determine whether a system under test satisfies requirements or works correctly \cite{comer}. For our function $f(a, b, c)$, a test case could be \textit{a = 1, b = 0, c = 1}.

\textbf{t-way combination} is a set of t input values, one from each of t input parameters \cite{comer}. For our function $f(a, b, c)$, a 2-way combination could be \textit{a = 1, b = 0}.

\textbf{t-way covering array} is a set of t-way combinations that covers all t-way combinations of input values. It is used to ensure that all combinations of input values are tested at least once. Fig. \ref{fig:CovArray} shows a covering array for $f(a, b, c)$ with 2-way combinations.

\begin{figure}
\centering
\begin{tabular}{|c|c|c|}
    \hline
    a & b & c \\
    \hline
    0 & 0 & 0 \\
    0 & 1 & 1 \\
    1 & 0 & 1 \\
    1 & 1 & 0 \\
    \hline
\end{tabular}
\caption{Covering array for $f(a, b, c)$ with 2-way combinations. Notice, that each 2 columns contain all possible combinations of 0's and 1's.}
\label{fig:CovArray}
\end{figure}

\subsection{Other Testing Techniques}

\textbf{Fuzz Testing} is a technique involving providing invalid, unexpected, or random data as input to a computer program. The program is then monitored for exceptions such as crashes or failing built-in code assertions, or for finding potential memory leaks \cite{Fuzz}.

\textbf{Boundary Value Analysis (BVA)} is a software testing technique in which tests are designed to include representatives of boundary values. It is based on the idea that input values at the extreme ends of the input domain are more likely to cause errors in the system \cite{BVA}.

\textbf{Property-based Testing} is a software testing technique that involves checking whether a system satisfies a set of properties, which are logical assertions about the system's behavior. It is a form of black-box testing, where the tester specifies the properties that the system should satisfy, but does not specify how the system should satisfy them \cite{Hypothesis}. \textit{Metamorphic Testing} is a form of property-based testing.

\subsection{General Testing Terms}

\textbf{Oracle Problem} is a problem in software testing where the tester does not know the correct output for a given input, and thus cannot determine if the system under test has passed or failed the test \cite{FELDERER20161}.

\textbf{System Model} is a model of the system under test (SUT) and/or its environment built from informal requirements, existing specification documents, or the SUT itself. It is used as input for automated test generation \cite{FELDERER20161}.

\textbf{Test Suite} is a collection of test cases that are used to test a software program to show that it has some specified set of behaviors.

\subsection{Cybersecurity}

\textbf{Cybersecurity} is the protection of computer systems and networks from information disclosure, theft of or damage to their hardware, software, or electronic data, as well as from the disruption or misdirection of the services they provide \cite{FELDERER20161}.

\textbf{Cybersecurity Testing} is for testing software system requirements related to security properties like confidentiality, integrity, availability, authentication, authorization, and non-repudiation. It involves techniques such as penetration testing, fuzz testing, static and dynamic analysis to identify vulnerabilities \cite{FELDERER20161}.

\subsection{Test Metrics}

\textbf{Test Metrics} are used to measure the effectiveness of the testing process. They are used to monitor the progress of testing, assess the quality of the software, and identify areas for improvement.

\textbf{Test Coverage} is a measure used to describe the degree to which the source code of a program is executed when a particular test suite runs. Some categorizations of test coverage provided below:

\begin{itemize}
    \item \textbf{Statement Coverage} measures the number of statements in the source code that have been executed by a test suite.
    \item \textbf{Branch Coverage} measures the number of if conditions, jumps, etc., in the source code that have been executed by a test suite.
    \item \textbf{Weak Coverage} means that the test suite has executed at least one of the possible paths through the code.
    \item \textbf{Strong Coverage} means that the test suite has executed all possible paths through the code.
\end{itemize}


\section{Structure of the Thesis}

This thesis is organized as follows:

\begin{itemize}
    \item \textbf{Introduction}: Provides an overview of the research problem, objectives, and proposed methodology.
    \item \textbf{Literature Review}: Surveys the existing literature on software testing techniques, cybersecurity, and the principles of metamorphic and Combinatorial Testing.
    \item \textbf{Methodology}: Details the proposed approach, including the integration of metamorphic and Combinatorial Testing, and the semi-automated test case generation process.
    \item \textbf{Implementation}: Describes the implementation of the proposed methodology and presents the results of empirical experiments conducted to evaluate its efficacy.
    \item \textbf{Evaluation and Discussion}: Analyzes the findings of the experiments and discusses their implications for software testing and cybersecurity.
    \item \textbf{Conclusion}: Summarizes the key findings of the research and outlines potential avenues for future research in this area.
\end{itemize}

\chapter{Literature Review}
\label{chap:lr}

% Cybersecurity threats are evolving as technology advances and new vulnerabilities are created, requiring organizations to continuously assess and address these vulnerabilities to maintain a secure environment.

Cybersecurity threats require innovative testing methods to ensure the reliability and security of the software. This literature review delves into the current state-of-the-art in combinatorial testing (CT) and metamorphic testing (MT) for cybersecurity, explores the integration of CT and MT, and assesses the efficiency of these methods working together and their potential applications in cybersecurity.

\section{Criteria and Process}

The selection and analysis of literature for this review were guided by specific criteria and a structured process. Search keywords included "Combinatorial Testing in Cybersecurity," "Metamorphic Testing in Cybersecurity," and "Integration of CT and MT in Cybersecurity." The search was conducted across multiple academic databases, including IEEE Xplore, ACM Digital Library, and Google Scholar. Articles were chosen based on the number of citations, the date of publication, the popularity of the journal, and relevance to the topic. The method involved a critical review of selected articles, focusing on their methodologies, findings, and implications in the context of cybersecurity testing.


\section{Combinatorial Testing in Cybersecurity}

Within the cybersecurity domain, combinatorial testing has been effectively utilized, particularly in generating test cases aimed at identifying vulnerabilities like SQL injection \cite{Simos2019SQL} and XSS injections \cite{Garn2014XSS}. The focus of both studies was primarily on a singular type of vulnerability. These investigations have demonstrated that combinatorial testing is both rapid and efficient in uncovering vulnerabilities. However, it is important to note that the scope of these studies is somewhat restricted in terms of the range of technologies evaluated.

\section{Metamorphic Testing in Cybersecurity}

The literature on Metamorphic Testing (MT) in cybersecurity provides insightful perspectives on its application and effectiveness. Segura \textit{et al.} highlight the fundamental concept of MT to overcome the oracle problem in software testing, where the correctness of the output is ambiguous \cite{CybersecurityMT}. This aspect is particularly crucial in cybersecurity, where accurate results are essential but often difficult to define.

In their exploration of MT's application to cybersecurity, researchers have demonstrated its efficacy in detecting hidden bugs in vital applications. A notable example is the testing of compilers and code obfuscators, where traditional testing methods may fail \cite{CybersecurityMT} \cite{GLSLFuzz}. The use of MT in such scenarios underlines its ability to handle complex and critical software systems.

The development and implementation of the METRIC framework \cite{ChenPoon2016} mark a significant advancement in MT. This approach systematically identifies metamorphic relations (MRs) through a category-choice framework, simplifying the process of MR identification which was previously manual and ad hoc. This systematic approach is particularly advantageous in cybersecurity, where the generation of reliable test oracles is challenging.

MT's role in cybersecurity extends to various practical applications. It facilitates negative testing and the evaluation of security-related functionalities, often problematic with conventional testing methods. A case in point is the use of MT in identifying vulnerabilities such as the Heartbleed bug in OpenSSL, demonstrating its ability to handle complex real-world cybersecurity issues \cite{OpenSSLMT}.

\section{Integration of Combinatorial and Metamorphic Testing}

The combination of Combinatorial Testing (CT) and Metamorphic Testing (MT) is explored by Niu \textit{et al.} \cite{comer} in an innovative approach to software testing. Their study tackles the challenge in CT related to the creation of automated test oracles, a problem that often leads to manual and error-prone processes in testing.

Niu \textit{et al.} introduce a new method called COMER, which blends MT with CT. This method focuses on forming pairs of test cases, known as Metamorphic Groups (MGs), based on specific metamorphic relations (MRs). These MGs allow for automatic checking of test results by verifying if the outputs go against the MRs.

\section{Conclusion}

The exploration of Combinatorial Testing (CT) and Metamorphic Testing (MT) within cybersecurity, as outlined in this literature review, lays the groundwork for advancing the field of software testing. Although these methodologies have been examined separately in various contexts, their integration, especially in addressing cybersecurity challenges, opens new avenues for research. The innovative approach of combining CT and MT, as proposed by Niu \textit{et al.} \cite{comer}, demonstrates a promising direction towards automating and enhancing the accuracy of testing processes. This research seeks to build on these foundational studies, focusing on the implications and practical applications of integrating CT and MT in the domain of cybersecurity, a critical area that has not been fully explored in the existing literature.

\newpage


% \begin{longtable}{c|c}
% \caption[This is the title I want to appear in the List of Tables]{Simulation Parameters} \label{table:secsimulation_params} \\
% \hline
% A & B  \\
% \hline
% \endfirsthead
% \multicolumn{2}{@{}l}{} \\
% \hline
% A & B \\
% \hline
% \endhead
% \hline
%  \textbf{Parameter} & \textbf{Value}\\
%  \hline
%  Number of vehicles & $|\mathcal{V}|$\\
%  \hline
%  Number of RSUs & $|\mathcal{U}|$\\
%  \hline
%  RSU coverage radius & 150 m\\
%  \hline
%  V2V communication radius & 30 m\\
%  \hline
%  Smart vehicle antenna height & 1.5 m\\
%  \hline
%  RSU antenna height & 25 m\\
%  \hline
%  Smart vehicle maximum speed & $v_{max}$ m/s\\
%  \hline
%  Smart vehicle minimum speed & $v_{min}$ m/s\\
%  \hline
%  Common smart vehicle cache capacities & $[50, 100, 150, 200, 250]$ mb\\
%  \hline
%  Common RSU cache capacities & $[5000,1000,1500,2000,2500]$ mb\\
%  \hline
%  Common backhaul rates & $[75, 100, 150]$ mb/s\\
%  \hline
% \end{longtable}

% \begin{figure}[hbt]
% \centering
% \includegraphics[]{figs/inno.png}
% \caption{One kernel at $x_s$ (\emph{dotted kernel}) or two kernels at
% $x_i$ and $x_j$ (\textit{left and right}) lead to the same summed estimate
% at $x_s$. This shows a figure consisting of different types of
% lines. Elements of the figure described in the caption should be set in
% italics, in parentheses, as shown in this sample caption.}
% \label{fig:secex}
% \end{figure}

% This description implies several essential properties of the task at hand:
% \begin{enumerate}
%     \item Watermark must contain all necessary information, but still, be placeable and recognizable even on smaller images. The produced watermark must be compact but have the possibility to store enough information. 
%     \item To prevent easy tampering, the watermark must be invisible to the naked eye (and, preferably, to basic image parsing tools). If malefactor does not know about the existence of watermark, they might not even try to remove it and disable it. 
% \end{enumerate}

%!TEX root = ../thesis.tex

\chapter{Methodology}
\label{ch:met}

This study employs the COMER methodology for the generation of test cases.
In order to assess the efficacy of the framework and conduct comparative analyses with other existing frameworks, I undertook the task of implementing it from scratch using the Python programming language.
The initial implementation by Niu \textit{et al.} \cite{comer} was coded in Java.
However, this implementation presented certain limitations; the code was compiled into a jar file, making it challenging to extend or modify functionalities.
Consequently, I opted to develop my own implementation in Python, making necessary modifications to facilitate experimentation with different datasets and enhancing the usability of the framework.
Python was chosen due to its user-friendly syntax and the availability of extensive libraries for data manipulation and visualization.


\section{COMER Framework}\label{sec:comer-framework}

The COMER framework serves the purpose of generating test cases, leveraging both traditional Combinatorial Testing (CT) methods and incorporating metamorphic relations during the generation process.
Upon receiving inputs including a set of metamorphic relations, a set of constraints, and abstract input parameters, COMER generates test cases that adhere to the specified constraints while satisfying certain metamorphic relations.

The framework operates in two primary stages:

\subsection{Abstract Test Case Generation}\label{subsec:abstract-test-case-generation}

The algorithm for abstract test case generation encompasses two main pathways.
At each step, there exists a probability distribution determining the choice between two flows.
In the first flow, the algorithm functions akin to a conventional CT generation algorithm, employing a greedy approach to generate test cases until achieving the desired t-way coverage.
Conversely, the second flow involves the selection of a previously generated test case, followed by the generation of subsequent test cases utilizing metamorphic relations.

\subsection{Concrete Test Case Generation}\label{subsec:concrete-test-case-generation}

Subsequently, the generated abstract test cases are mapped to concrete ones.
For test cases generated via CT methods, the concrete test case is produced by randomly selecting a value from the domain corresponding to the abstract value.
Conversely, for test cases generated through metamorphic relations, the framework utilizes the specified metamorphic relation to generate subsequent test cases.

\begin{figure}[hbt]
    \centering
    \includegraphics[]{figs/test_cases.png}
    \caption{Example of generating test cases for \textit{FindClosest} function. Note that $T1$ and $T2$ are parameters generated by utilizing MR.}
    \label{fig:secex}
\end{figure}


\section{Limitations}\label{sec:limitations}

Despite the advancements made in the original COMER implementation, certain limitations were identified.
Firstly, the reliance on a greedy approach for generating Combinatorial Testing (CT) test cases posed a notable drawback.
While this method can increase the speed of generation process, it may result in an excessive number of test cases, potentially compromising efficiency and resource utilization.
Secondly, the focus of the implementation solely on metamorphic relations with a single input and a single follow-up test case restricted its applicability to scenarios involving more complex MRs. Thirdly, the lack of provision for generating test cases for functions beyond those already supported in the implementation posed a significant constraint.

In my implementation, efforts were made to mitigate some of these limitations.
To overcome the limitation of supporting only a limited range of functions, I worked to improve the framework by allowing the generation of test cases for any function.
Additionally, in response to concerns regarding the effectiveness of the greedy approach, provisions were made to incorporate alternative algorithms for CT test case generation, thereby offering users flexibility in selecting the most suitable approach.
However, it is important to note that the limitation pertaining to MRs with multiple inputs and follow-up test cases remains unaddressed in my implementation, representing an area for potential future research and improvement.

\chapter{Реализация}
\label{chap:impl}

Основная функция, отвечающая за генерацию тестовых примеров, выглядит следующим образом:

\begin{lstlisting}[label={lst:test-case1}]
def __next__(self) -> OrderedDict:
	r = random.random()
	if r > self._mr_probability and self.generated_cases:
		case_pre = random.choice(self.generated_cases)
		solution = self.csp_solver(case_pre)
		if solution:
			self.add_testcase_to_tested(solution)
			return solution

	return OrderedDict(super().__next__())

\end{lstlisting}

Функция придерживается основополагающих принципов методологии COMER.
В частности, на основе заданной вероятности, \verb|self._mr_probability|, функция выбирает подход, основанный на комбинаторном тестировании (КT) или метаморфическом тестировании (MT).
В КT для итеративной генерации тестовых примеров используется решатель задачи удовлетворения ограничений (Constraint Satisfaction Problem, CSP).
В MT он выбирает существующий тестовый пример (\verb|case_pre|) из предоставленного пула (\verb|self.generated_cases|) и генерирует последующие тестовые примеры с помощью CSP-решателя.

\section{Использование методологии}

Пример демонстрирует, как использовать фреймворк для тестирования веб-приложения. Предположим, мы хотим протестировать веб-приложение, которое позволяет пользователям добавлять закладки и смарт-теги в свои учетные записи. Мы можем определить набор параметров для наших тестовых случаев следующим образом:

\begin{lstlisting}[label={lst:test-case2}]
params = OrderedDict(
	{
		"highlight": [0, 1],
		"status_bar": [0, 1],
		"bookmarks": [0, 1],
		"smart_tags": [0, 1],
	}
)
\end{lstlisting}

Ограничения определяются как функции, которые принимают тестовый пример на вход и выдают True или False, указывая, должны ли мы включить конкретный тестовый пример в наш домен. Метаморфные связи определяются с использованием тестового случая в качестве входа и нового последующего действия в качестве выхода. В этом конкретном примере мы не определяем дополнительные МС для генерации конкретных тестовых случаев для простоты.

\begin{lstlisting}[label={lst:test-case3}]
def constraint(highlight, status_bar, **kwargs):
	return highlight == status_bar

def mr(bookmarks, smart_tags, **kwargs):
	return OrderedDict({
		"bookmarks": smart_tags,
		"smart_tags": bookmarks,
		**kwargs
	})
\end{lstlisting}

После того как домен задан, мы можем создавать абстрактные тестовые примеры. В этом примере мы используем их в качестве входных данных для \textit{Pytest} для создания параметризованных тестовых примеров.

\begin{lstlisting}[label={lst:test-case4}]
@pytest.mark.parametrize("testcase", Comer(params, constraint, MR(mr)))
def test_simple(testcase: OrderedDict):
	assert testcase["highlight"] == testcase["status_bar"]
\end{lstlisting}

\chapter{Выводы}
\label{chap:conclusion}

В данной дипломе представлено комплексное исследование интеграции комбинаторного тестирования (КТ) и метаморфического тестирования (МТ) для автоматизированного тестирования безопасности веб-приложений. Предложенная реализация фреймворка COMER объединяет сильные стороны КТ и МТ для генерации тестовых примеров, которые могут эффективно выявлять уязвимости в сложном программном обеспечении.

Оценка фреймворка COMER показала его эффективность в генерации тестовых примеров, способных обнаружить уязвимости в веб-приложениях. Результаты показывают, что фреймворк COMER может генерировать тестовые примеры более эффективно, чем традиционные методы КТ, с более высокой частотой обнаружения и аналогичным временем генерации. Однако предварительные тесты текущей реализации показали, что время выполнения в некоторых случаях удваивается по сравнению с традиционным фреймворком CT из-за увеличения количества генерируемых тестовых примеров.

Исследование также продемонстрировало потенциал фреймворка COMER для стандартизации в виде фреймворка или набора инструментов для автоматизированного тестирования безопасности. Способность фреймворка эффективно выявлять уязвимости и генерировать тестовые примеры делает его ценным инструментом для специалистов по кибербезопасности, стремящихся повысить уровень защиты своих приложений.

Однако исследование также выявило необходимость проведения дополнительных исследований в области систематической генерации метаморфических отношений (MR) для сложного программного обеспечения кибербезопасности. Ручное создание MRs для каждого конкретного тестового случая - трудоемкий и длительный процесс, требующий специальных знаний и опыта. Разработка методологий или инструментов, позволяющих автоматизировать процесс создания MR, значительно повысит эффективность фреймворка COMER.



\printbibliography[heading=bibintoc,title={Список использованной литературы}]
\end{document}
