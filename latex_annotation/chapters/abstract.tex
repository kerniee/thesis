\begin{abstract}
% skip one line to make the abstract start with indent

Стремительное развитие программного обеспечения и его растущая интеграция в различные аспекты нашей повседневной жизни привели к появлению проблем в обеспечении его безопасности. В результате слияния не ИТ и ИТ-индустрий увеличилось количество ошибок и уязвимостей в программном обеспечении, которые влекут за собой значительные риски для отдельных людей, организаций и общества в целом. Несмотря на постоянные усилия по улучшению кибербезопасности, вездесущность киберугроз и отсутствие единого надежного метода обеспечения корректности программного обеспечения по-прежнему остаются актуальными проблемами.

Данный дипломный проект направлен на решение критической потребности в повышении кибербезопасности путем изучения потенциала метаморфического тестирования и комбинаторного тестирования как эффективных методов выявления и устранения уязвимостей программного обеспечения. Цель исследования - оценить совместное использование этих методов и оценить их эффективность, действенность и всесторонний охват при обнаружении дефектов безопасности.

Исследовательский проект позволил получить значительные результаты, продемонстрировав, что интеграция методов метаморфического и комбинаторного тестирования значительно повышает эффективность и качество тестирования безопасности программного обеспечения.

\end{abstract}
