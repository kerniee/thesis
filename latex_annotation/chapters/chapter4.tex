\chapter{Реализация}
\label{chap:impl}

Основная функция, отвечающая за генерацию тестовых примеров, выглядит следующим образом:

\begin{lstlisting}[label={lst:test-case1}]
def __next__(self) -> OrderedDict:
	r = random.random()
	if r > self._mr_probability and self.generated_cases:
		case_pre = random.choice(self.generated_cases)
		solution = self.csp_solver(case_pre)
		if solution:
			self.add_testcase_to_tested(solution)
			return solution

	return OrderedDict(super().__next__())

\end{lstlisting}

Функция придерживается основополагающих принципов методологии COMER.
В частности, на основе заданной вероятности, \verb|self._mr_probability|, функция выбирает подход, основанный на комбинаторном тестировании (КT) или метаморфическом тестировании (MT).
В КT для итеративной генерации тестовых примеров используется решатель задачи удовлетворения ограничений (Constraint Satisfaction Problem, CSP).
В MT он выбирает существующий тестовый пример (\verb|case_pre|) из предоставленного пула (\verb|self.generated_cases|) и генерирует последующие тестовые примеры с помощью CSP-решателя.

\section{Использование методологии}

Пример демонстрирует, как использовать фреймворк для тестирования веб-приложения. Предположим, мы хотим протестировать веб-приложение, которое позволяет пользователям добавлять закладки и смарт-теги в свои учетные записи. Мы можем определить набор параметров для наших тестовых случаев следующим образом:

\begin{lstlisting}[label={lst:test-case2}]
params = OrderedDict(
	{
		"highlight": [0, 1],
		"status_bar": [0, 1],
		"bookmarks": [0, 1],
		"smart_tags": [0, 1],
	}
)
\end{lstlisting}

Ограничения определяются как функции, которые принимают тестовый пример на вход и выдают True или False, указывая, должны ли мы включить конкретный тестовый пример в наш домен. Метаморфные связи определяются с использованием тестового случая в качестве входа и нового последующего действия в качестве выхода. В этом конкретном примере мы не определяем дополнительные МС для генерации конкретных тестовых случаев для простоты.

\begin{lstlisting}[label={lst:test-case3}]
def constraint(highlight, status_bar, **kwargs):
	return highlight == status_bar

def mr(bookmarks, smart_tags, **kwargs):
	return OrderedDict({
		"bookmarks": smart_tags,
		"smart_tags": bookmarks,
		**kwargs
	})
\end{lstlisting}

После того как домен задан, мы можем создавать абстрактные тестовые примеры. В этом примере мы используем их в качестве входных данных для \textit{Pytest} для создания параметризованных тестовых примеров.

\begin{lstlisting}[label={lst:test-case4}]
@pytest.mark.parametrize("testcase", Comer(params, constraint, MR(mr)))
def test_simple(testcase: OrderedDict):
	assert testcase["highlight"] == testcase["status_bar"]
\end{lstlisting}
