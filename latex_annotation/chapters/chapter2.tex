\chapter{Обзор литературы}
\label{chap:background}

%!TEX root = ../thesis.tex

Угрозы кибербезопасности требуют инновационных методов тестирования для обеспечения надежности и безопасности программного обеспечения. В данном обзоре литературы рассматриваются современные достижения в области комбинаторного тестирования (КТ) и метаморфического тестирования (МТ) для кибербезопасности, изучается интеграция КТ и МТ, оценивается эффективность совместной работы этих методов и их потенциальное применение в кибербезопасности.

\section{Комбинаторное тестирование в кибербезопасности}\label{sec:combinatorial-testing-in-cybersecurity}

В области кибербезопасности комбинаторное тестирование было использовано, в частности, для создания тестовых примеров, направленных на выявление таких уязвимостей, как SQL-инъекции \cite{Simos2019SQL} и XSS-инъекции \cite{Garn2014XSS}. В обоих исследованиях основное внимание уделялось одному типу уязвимости. Эти исследования показали, что комбинаторное тестирование быстро и эффективно выявляет уязвимости. Однако важно отметить, что рамки этих исследований несколько ограничены с точки зрения спектра оцениваемых технологий.


\section{Метаморфическое тестирование в кибербезопасности}\label{sec:metamorphic-testing-in-cybersecurity}

Литература, посвященная метаморфическому тестированию (МТ) в кибербезопасности, подробно рассматривает его применение и эффективность. Segura \textit{и др.} выделяют ключевую роль МТ в решении проблемы оракула при тестировании программного обеспечения, когда правильность результата неочевидна. \cite{CybersecurityMT}. Это особенно важно в кибербезопасности, где точные результаты критически важны, но зачастую их трудно определить.

Исследуя возможности применения МТ в кибербезопасности, исследователи продемонстрировали его эффективность в обнаружении скрытых ошибок в жизненно важных приложениях. Ярким примером является тестирование компиляторов и обфускаторов кода, где традиционные методы тестирования могут оказаться неэффективными \cite{CybersecurityMT} \cite{GLSLFuzz}. Использование МТ в таких сценариях подчеркивает его способность работать со сложными и критическими программными системами.

Разработка и внедрение фреймворка METRIC \cite{ChenPoon2016} ознаменовали собой значительный прогресс в области МТ. Этот подход систематически определяет метаморфические отношения (МО) с помощью системы выбора категорий, упрощая процесс идентификации МО, который ранее был ручным и несистематическим. Этот систематический подход особенно полезен для кибербезопасности, где создание надежных тестовых оракулов является сложной задачей. Якушева и др. показали, что выведение MR можно еще больше упростить, используя специфические для данной области рекомендации \cite{MetaRU}.

\section{Интеграция комбинаторного и метаморфического тестирования}\label{sec:integration-of-combinatorial-and-metamorphic-testing}

Сочетание комбинаторного тестирования (КТ) и метаморфического тестирования (МТ) рассматривается в работе Niu \textit{et al.} \cite{comer} в инновационном подходе к тестированию программного обеспечения. В их исследовании рассматривается проблема, связанная с созданием автоматизированных тестовых оракулов, которая часто приводит к ручному и подверженному ошибкам процессу тестирования.

Niu \textit{и др.} представляют новый метод под названием COMER, который сочетает MT с CT. Этот метод фокусируется на формировании пар тестовых случаев, известных как метаморфические группы (МГ), на основе определенных метаморфических отношений (МО). Эти MG позволяют автоматически проверять результаты тестирования, проверяя, соответствуют ли выходные данные MR.

\newpage
