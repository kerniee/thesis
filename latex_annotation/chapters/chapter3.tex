\chapter{Методология}
\label{chap:met}

%!TEX root = ../thesis.tex

Это исследование использует методологию COMER для генерации тестов.
Чтобы оценить эффективность этой методологии и провести сравнительный анализ с другими существующими методологиями, я взял на себя задачу реализовать её с нуля с использованием языка программирования Python.

\section{Методология COMER}\label{sec:comer-framework}

Методология COMER предназначена для генерации тестов, используя как традиционные методы комбинаторного тестирования (КТ), так и добавляя метаморфные связи в процессе генерации.
После получения входных данных, включая набор метаморфных связей, набор ограничений и абстрактные входные параметры, COMER генерирует тесты, которые соответствуют указанным ограничениям и удовлетворяют определённым метаморфным связям (МС).

\section{Ограничения}\label{sec:limitations}

Несмотря на достижения в первоначальной реализации COMER, были выявлены определённые ограничения.
Во-первых, использование жадного подхода для генерации тестов КТ являлось заметным недостатком.
Хотя этот метод может повысить скорость процесса генерации, он может привести к избыточному количеству тестов, что потенциально снижает эффективность и использование ресурсов.
Во-вторых, ориентация реализации исключительно на метаморфные связи с одним входным и одним последующим тестом ограничивала её применимость к сценариям с более сложными МС. В-третьих, отсутствие возможности генерировать тесты для функций, не поддерживаемых в реализации, представляло значительное ограничение.

В моей реализации были предприняты усилия по смягчению некоторых из этих ограничений.
Чтобы преодолеть ограничение поддержки только ограниченного числа функций, я работал над улучшением методологии, позволяя генерировать тесты для любой функции.
Кроме того, в ответ на обеспокоенность по поводу эффективности жадного подхода были предусмотрены альтернативные алгоритмы для генерации тестов КТ, предоставляя пользователям возможность выбрать наиболее подходящий подход.
Однако важно отметить, что ограничение, касающееся МС с множественными входами и последующими тестами, остаётся неустранённым в моей реализации, представляя собой область для потенциальных будущих исследований и улучшений.
