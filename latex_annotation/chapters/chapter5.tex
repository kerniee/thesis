\chapter{Выводы}
\label{chap:conclusion}

В данной дипломе представлено комплексное исследование интеграции комбинаторного тестирования (КТ) и метаморфического тестирования (МТ) для автоматизированного тестирования безопасности веб-приложений. Предложенная реализация фреймворка COMER объединяет сильные стороны КТ и МТ для генерации тестовых примеров, которые могут эффективно выявлять уязвимости в сложном программном обеспечении.

Оценка фреймворка COMER показала его эффективность в генерации тестовых примеров, способных обнаружить уязвимости в веб-приложениях. Результаты показывают, что фреймворк COMER может генерировать тестовые примеры более эффективно, чем традиционные методы КТ, с более высокой частотой обнаружения и аналогичным временем генерации. Однако предварительные тесты текущей реализации показали, что время выполнения в некоторых случаях удваивается по сравнению с традиционным фреймворком CT из-за увеличения количества генерируемых тестовых примеров.

Исследование также продемонстрировало потенциал фреймворка COMER для стандартизации в виде фреймворка или набора инструментов для автоматизированного тестирования безопасности. Способность фреймворка эффективно выявлять уязвимости и генерировать тестовые примеры делает его ценным инструментом для специалистов по кибербезопасности, стремящихся повысить уровень защиты своих приложений.

Однако исследование также выявило необходимость проведения дополнительных исследований в области систематической генерации метаморфических отношений (MR) для сложного программного обеспечения кибербезопасности. Ручное создание MRs для каждого конкретного тестового случая - трудоемкий и длительный процесс, требующий специальных знаний и опыта. Разработка методологий или инструментов, позволяющих автоматизировать процесс создания MR, значительно повысит эффективность фреймворка COMER.
