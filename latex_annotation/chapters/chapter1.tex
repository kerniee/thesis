\chapter{Введение}
\label{chap:intro}

%!TEX root = ../thesis.tex

\section{Мотивация}\label{sec:motivation}

В последние годы быстрая эволюция разработки программного обеспечения привела к значительному росту числа ошибок и уязвимостей в программном обеспечении.
Растущее число общих уязвимостей и уязвимых мест (CVE), о которых сообщается каждый год, отражает растущие проблемы, возникающие при обеспечении безопасности и надежности программных систем \cite{CVE}.

Традиционные подходы к тестированию программного обеспечения часто оказываются недостаточными для эффективного обнаружения и устранения уязвимостей.
С принятием практики непрерывной интеграции в качестве стандарта в отрасли возрос спрос на автоматизированные методы тестирования, которые могут легко интегрироваться в рабочий процесс разработки.
Главная цель - выявить и устранить ошибки и уязвимости на самых ранних этапах жизненного цикла разработки программного обеспечения.

\section{Предлагаемый подход}\label{sec:proposed-approach}

В данном дипломе предлагается методология, объединяющая две различные техники тестирования: Метаморфическое тестирование и Комбинаторное тестирование.
В предлагаемом подходе используется полуавтоматизированная методология генерации тестовых примеров.
Используя принципы метаморфического и комбинаторного тестирования, мы стремимся упростить процесс тестирования, обеспечивая при этом адекватное покрытие тестируемой системы.
Этот гибридный подход позволяет найти баланс между ручным вмешательством и автоматизированным тестированием, оптимизируя эффективность и результативность процесса тестирования.
